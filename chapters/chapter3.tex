\chapter{现代密码学}

\section{RSA}

\subsection{RSA}

RSA是一种非对称加密(asymmetric cryptography)算法,在1977年由Ron Rivest、Adi ShamirLeonard Adleman一起提出,RSA就是他们三人姓氏的首字母。\\

所谓非对称加密,是指在网络通信中双方各有一对密钥,其中公钥(public key)被公开给外界,用于加密;私钥(private key)只有自己知道,用于解密。\\

例如Alice的一组密钥为$ (P_A, S_A) $,Bob的一组密钥为$ (P_B, S_B) $。对一段消息$ M $加密和解密的操作是互逆的:

\vspace{-1cm}

\begin{align*}
    M & = S_A(P_A(M)) \\
    M & = P_A(S_A(M))
\end{align*}

如果Bob想要给Alice发送消息,Bob首先使用Alice的公钥$ P_A $对消息$ M $进行加密得到密文$ C $:

\vspace{-1cm}

\begin{align*}
    C = P_A(M)
\end{align*}

Bob将密文$ C $发送给Alice,Alice使用自己的私钥$ S_A $对密文$ C $进行解密得到原文$ M $:

\vspace{-1cm}

\begin{align*}
    S_A(C) = S_A(P_A(M)) = M
\end{align*}

\vspace{0.5cm}

\subsection{公钥/私钥生成}

RSA的安全性依赖于大整数的质因数分解,也就是对于两个大素数$ p $和$ q $而言,计算它们的乘积$ pq $很容易,但是从积$ pq $分解出$ p $和$ q $是个公认的数学难题。\\

RSA公钥和私钥生成的过程如下:

\begin{enumerate}
    \item 随机选择两个大素数(超过100位),为了简化说明,这里采用较小的素数$ p = 41 $、$ q = 59 $

    \item 计算$ p $和$ q $的乘积$ n = pq = 2419 $

    \item 计算$ p - 1 $与$ q - 1 $的乘积$ \phi(n) = (p-1)(q-1) = 40 * 58 = 2320 $

    \item 选择一个小奇数$ e $,使得$ gcd(e, \phi(n)) = 1$,例如$ e = 3 $

    \item 计算$ d $,使得$ d * e \equiv 1\ (\text{mod}\ \phi(n)) $。当$ d = 1547 $时,$ 1547 * 3\ \text{mod}\ 2320 = 1 $

    \item 公钥$ P = (e, n) = (3, 2419) $

    \item 私钥$ S = (d, n) = (1547, 2419) $

    \item 对消息$ M $加密的过程为$ P(M) = M^e\ (\text{mod}\ n) = M^3\ (\text{mod}\ 2419) $

    \item 对消息$ M $解密的过程为$ S(M) = M^d\ (\text{mod}\ n) = M^{1547}\ (\text{mod}\ 2419) $
\end{enumerate}

\vspace{0.5cm}

\subsection{加密}

假设Alice的公钥为$ (9, 2419) $,想要将消息“PACE”发送给Bob。\\

首先将消息中的英文字母转换为对应的编码:

\begin{itemize}
    \item PA = 1500
    \item CE = 0204
\end{itemize}

\begin{table}[H]
    \centering
    \setlength{\tabcolsep}{5mm}{
        \begin{tabular}{|c|c|c|c|}
            \hline
            \textbf{Letter} & \textbf{Code} & \textbf{Letter} & \textbf{Code} \\
            \hline
            A               & 00            & B               & 01            \\
            \hline
            C               & 02            & D               & 03            \\
            \hline
            E               & 04            & F               & 05            \\
            \hline
            G               & 06            & H               & 07            \\
            \hline
            I               & 08            & J               & 09            \\
            \hline
            K               & 10            & L               & 11            \\
            \hline
            M               & 12            & N               & 13            \\
            \hline
            O               & 14            & P               & 15            \\
            \hline
            Q               & 16            & R               & 17            \\
            \hline
            S               & 18            & T               & 19            \\
            \hline
            U               & 20            & V               & 21            \\
            \hline
            W               & 22            & X               & 23            \\
            \hline
            Y               & 24            & Z               & 25            \\
            \hline
        \end{tabular}
    }
\end{table}

分别对PA和CE进行加密:

\begin{itemize}
    \item $ P(1500) = 1500^9\ (\text{mod}\ 2419) = 1655 $
    \item $ P(0204) = 204^9\ (\text{mod}\ 2419) = 1639 $
\end{itemize}

最终形成的密文为1655 1639。

\newpage