\chapter{数论}

\section{最大公约数 / 最小公倍数}

\subsection{最大公约数(GCD, Greatest Common Divisor)}

两个整数$ a $和$ b $的最大公约数$ gcd(a, b) $为能够同时整除$ a $和$ b $的最大整数。\\

例如:

\begin{itemize}
    \item $ gcd(24, 36) = 12 $
    \item $ gcd(17, 22) = 1 $
    \item $ gcd(500, 128) = 4 $
\end{itemize}

计算最大公约数的算法包括递归方法和欧几里得(Euclidean)算法/辗转相除法。\\

\mybox{最大公约数}

\begin{lstlisting}[language=Python]
def gcd(a, b):
    if b == 0:
        return a
    return gcd(b, a % b)


def euclid_gcd(a, b):
    while b != 0:
        remainder = a % b
        a = b
        b = remainder
    return a
\end{lstlisting}

\vspace{0.5cm}

\subsection{最小公倍数(LCD, Least Common Multiple)}

两个整数$ a $和$ b $的最小公倍数$ lcm(a, b) $为能够同时被$ a $和$ b $整除的最小整数。\\

例如:

\begin{itemize}
    \item $ lcm(24, 36) = 72 $
    \item $ lcm(17, 22) = 374 $
    \item $ lcm(500, 128) = 16000 $
\end{itemize}

\vspace{0.5cm}

\mybox{最小公倍数}

\begin{lstlisting}[language=Python]
def lcm(a, b):
    return a * b // gcd(a, b)
\end{lstlisting}

\newpage

\section{同余定理}

\subsection{模算数(Modular Arithmetic)}

当$ a \in \mathbb{Z} $、$ M \in \mathbb{Z^+} $,那么将$ a $除以$ m $的余数记为$ a\ \text{mod}\ m $。\\

例如:

\begin{itemize}
    \item 17 mod 5 = 2
    \item 2001 mod 101 = 82
    \item -10 mod 3 = -1
\end{itemize}

\vspace{0.5cm}

\subsection{同余定理(Congruence Theorem)}

当$ a \in \mathbb{Z} $、$ b \in \mathbb{Z} $、$ M \in \mathbb{Z^+} $,如果$ m $能够整除$ a - b $,那么就称$ a $和$ b $对模$ m $同余,记作$ a \equiv b\ (\text{mod}\ m) $。\\

因此,

\vspace{-1cm}

\begin{align}
    a \equiv b\ (\text{mod}\ m) \leftrightarrow a\ \text{mod}\ m \equiv b\ \text{mod}\ m
\end{align}

例如:

\begin{itemize}
    \item $ 17 \equiv 5\ (\text{mod}\ 6) $
    \item $ 17 \equiv 12\ (\text{mod}\ 5) $
    \item $ 24 \equiv 3\ (\text{mod}\ 7) $
\end{itemize}

\vspace{0.5cm}

当$ a \equiv b\ (\text{mod}\ m) $、$ c \equiv d\ (\text{mod}\ m) $,同余定理满足以下性质:

\begin{itemize}
    \item $ a + c \equiv b + d\ (\text{mod}\ m) $
    \item $ ac \equiv bd\ (\text{mod}\ m) $
\end{itemize}

\vspace{0.5cm}

\begin{tcolorbox}
    \mybox{Exercise}\\
    因为$ 7 \equiv 2\ (\text{mod}\ 5) $、$ 11 \equiv 1\ (\text{mod}\ 5) $。\\
    (a) $ 7 + 11\ (\text{mod}\ 5) = 2 + 1 \ (\text{mod}\ 5) = 3 $\\
    (b) $ 7 \cdot 11\ (\text{mod}\ 5) = 2 \cdot 1 \ (\text{mod}\ 5) = 2 $
\end{tcolorbox}

\vspace{0.5cm}

\begin{tcolorbox}
    \mybox{Exercise}\\
    (a) $ 7^{10} \text{mod}\ 5 = 2^{10} \text{mod}\ 5 = 4 $\\
    (b) $ 7^{100} \text{mod}\ 3 = 1^{100} \text{mod}\ 5 = 1 $
\end{tcolorbox}

\newpage